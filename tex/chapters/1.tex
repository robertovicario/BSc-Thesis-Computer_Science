\chapter{Introduction}

The detection of stress in workplace environments represents an innovative approach to promoting employee health and well-being, particularly within office settings. In this context, the use of machine learning offers an effective method for identifying early signs of stress, allowing organizations to intervene promptly to mitigate associated risks. Specifically, this study aims to examine whether the results obtained through unsupervised learning methods can be comparable to those derived from supervised approaches, as highlighted in the research *"Exploring Unsupervised Machine Learning Classification Methods for Physiological Stress Detection"* by Iqbal et al. (2022) \cite{iqbal2022exploring}.

\vspace{0.5cm}

The SWELL-KW dataset served as the primary resource for extracting data to train artificial intelligence models in detecting stress among employees. This chapter explores the data collection process, outlining the methodologies employed to ensure the quality of the dataset. Subsequently, the features extracted from the data and their associated labels are analyzed.

\vspace{0.5cm}

The analysis then proceeds with an in-depth examination of biomedical signals, which are assigned critical importance within the scope of the research. A distinction is made between physical and physiological signals, with a focused discussion on the specific signals utilized. Particularly noteworthy is heart rate, represented through the electrocardiogram (ECG), which plays a central role in stress detection studies. The exploration of these elements lays the foundational basis for understanding the principal machine learning methodologies in the context of heart rate variability (HRV).

\vspace{0.5cm}

Following this, the state-of-the-art research on stress detection using machine learning methods is presented. The methodologies and findings discussed in Chapter Four are of key importance in outlining the current state of research and in providing meaningful context for future investigations. These elements are especially relevant for defining the methodologies and comparing the results obtained in this project.

\vspace{0.5cm}

Finally, the adopted methodology is described, beginning with the technologies employed in the project's development and continuing through to the machine learning methods used for training and evaluating the results. Chapters Five and Six detail the project from both theoretical and practical perspectives, providing an in-depth analysis of the implementation process. In conclusion, the results obtained from the models developed through these methodologies are presented and compared with the current state-of-the-art. Lastly, guidelines are proposed for future research on stress detection using machine learning, leveraging the SWELL-KW dataset to analyze stress in office work environments.
